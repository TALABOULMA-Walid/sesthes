\documentclass{standalone}


\usepackage{tikz}
\usetikzlibrary{shapes,arrows}
\begin{document}


\tikzstyle{block} = [draw, rectangle, 
    minimum height=3em, minimum width=6em]
\tikzstyle{sum} = [draw, circle, node distance=1cm]
\tikzstyle{input} = [coordinate]
\tikzstyle{output} = [coordinate]
\tikzstyle{pinstyle} = [pin edge={to-,thin,black}]

% The block diagram code is probably more verbose than necessary
\begin{tikzpicture}[auto, node distance=2cm,>=latex']
    % We start by placing the blocks
    \node [input, name=input] {};
    \node [sum, right of=input,node distance=2cm] (sum) {};
    \node [block, right of=sum,node distance=3.5cm] (controller) {Controller};
    \node [block, right of=controller,
            node distance=4cm] (system) {Physical process};
    % We draw an edge between the controller and system block to 
    % calculate the coordinate u. We need it to place the measurement block. 
    \draw [->] (controller) -- node[name=u] {Input} (system);
    \node [output, right of=system, node distance = 3cm] (output) {};
    \node [block, below of=u] (measurements) {Sensor};

    % Once the nodes are placed, connecting them is easy. 
    \draw [draw,->] (input) -- node {Reference} (sum);
    \draw [->] (sum) -- node[text width=1.5cm,align=center] {Measured error} (controller);
    \draw [->] (system) -- node [name=y] {Output}(output);
    \draw [->] (y) |- (measurements);
    \draw [->] (measurements) -| node[pos=0.99] {$-$} 
        node [near end,text width=1.5cm,align=center] {Measured output} (sum);
\end{tikzpicture}

\end{document}

\end{document}