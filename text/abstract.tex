When designing complex cyber-physical systems, engineers have to integrate numerical models
from different
modeling environments in order to simulate the whole system and estimate its global
performances. If some parts of the system are physically available, it is possible to connect these
parts to the simulation in a Hardware-in-the-Loop (HiL) approach. In this case, the
simulation has to be performed in real-time where models execution consists in periodically
reacting to the real (physically available) components and providing periodic output updates. The
increase of requirements on the simulation accuracy and its validity domain requires more
complex models. Using such models, it becomes hard to ensure fast or real-time execution without using
multiprocessor architectures. FMI (Functional Mocked-up Interface), an increasingly common
standard for model exchange and co-simulation, offers
new opportunities for multi-core execution
of numerical models%, by enabling intra-model parallelization
. One goal of this thesis is the extraction of potential parallelism in a set of interconnected multi-rate models. We build on the RCOSIM approach that has been previously developed at IFP Energies nouvelles and which allows the parallelization of FMI models on multi-core processors. It is based on representing the co-simulation by a dependence graph model. In the first part of the thesis, improvements have been proposed to overcome the limitations of RCOSIM. In particular, we propose new algorithms in order to allow
handling models that exchange data at different
rates and schedule them on multi-core processors. Also, the improvements allow handling specific constraints such as mutual exclusion and real-time constraints. 
Second, we propose algorithms for the allocation and non preemptive scheduling of the dependence graphs, taking % cosimulation: accleration et depasser les limites, garantir le temps réel pour le HiL
into account their real-time, data dependence and allocation constraints. These algorithms aim at accelerating the execution of the co-simulation or ensuring its real-time execution in a HiL approach.
The proposed solutions have been tested on randomly generated dependence graphs and validated against an industrial use case which is an internal combustion engine co-simulation.
This thesis is part of a joint action IFP Energies nouvelles - Inria in which Inria brings its real-time systems experience to the numerical simulation challenges of IFP Energies nouvelles.