When designing complex cyber-physical systems, engineers have to integrate numerical models
from different
modeling environments in order to simulate the whole system and estimate its global
performances. If some parts of the system are already available, it is possible to connect these
real components to the simulation in a Hardware-in-the-Loop (HiL) approach. In this case, the
simulation has to be performed in real-time where models execution consists in periodically
reacting to inputs from the real components and providing numerical output updates. The
increase of requirements on the simulation accuracy and its validity domain requires more
complex models. Using such models, it becomes hard to ensure fast as well as real-time execution without using
multiprocessor architectures. FMI (Functional Mocked-up Interface), an increasingly common
standard for model exchange and co-simulation, oers
new opportunities for multi-core execution
of numerical models, by enabling intra-model parallelization. One objective of this thesis is to
define algorithms for extracting potential parallelism in a set of interconnected multi-rate models.
Another one consists in proposing algorithms for the allocation and scheduling of models, taking
into account their real-time, data dependencies and allocation constraints. Such algorithms aim to accelerate the execution of the co-simulation or ensure its real-time execution. Prior to this thesis,
an approach has been developed at IFPEN which allows the parallelization of FMI models on
multi-core processors. In the first part of the thesis, improvements have been proposed to overcome
the limitations of this approach. In particular, we propose new algorithms in order to allow
handling models that exchange data at different
rates and schedule them on multi-core processors.
The proposed algorithms allow also the optimization of the phase of analyzing the models structures and
their data dependencies in order to compute their schedule on the multi-core processor. Finally, the
proposed improvements allow more accurate measurements of the execution times of the models using a profiling technique and allow also handling some specific constraints like mutual exclusion constraints. This thesis is part of a joint action IFPEN - INRIA in which INRIA brings its
real-time systems experience to the numerical simulation challenges of IFPEN.