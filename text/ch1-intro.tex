
\chapter{\label{ch:1-intro}Introduction} 

\minitoc

This thesis deals with the parallelization of co-simulations of numerical models on multi-core architectures. In particular, it focuses on the acceleration and real-time execution of co-simulations through multi-core parallelization. Different research questions related to the aforementioned problems are studied. The work presented in this thesis represents a continuation of two PhD theses that have been previously conducted at IFP Energies nouvelles\footnote{www.ifpenergiesnouvelles.fr}. See \cite{faure:2011,benkhaled:2014_thesis}. This first chapter gives an introduction to the research topic of the thesis. First, we explain the general context of the studied research problems. Then, we briefly present the objectives and the contributions of the thesis. Finally, we give the structure of the thesis. 

\section{Context}

The number of computers has grown very fast in recent decades and today they are omnipresent. The most known kind of computers is general purpose computers that are used for human consumption. However, the vast majority of the computers around the world are less visible and are used for different purposes, mainly for controlling physical entities. These computers are called embedded systems. 

Systems that combine computational elements and physical processes are referred to as Cyber-Physical Systems (CPS) \cite{lee:2016}. The diversity of the involved disciplines makes the process of building CPS challenging, costly and time consuming. Therefore, applying appropriate methodologies that respond to challenges related to the design, the development and the validation of CPS, is a crucial requirement. In particular, enabling the prediction of the system's behavior before its deployment has the potential to reduce the risks, the cost and the needed effort. Simulation is an efficient way to achieve these requirements as it allows imitating the functioning of the system on a computer and assessing its design. System designers can then identify potential design flaws and correct them before deploying the system.

In order to perform the simulation, the system is first modeled. Traditionally, subparts of the system are modeled separately, and then integrated into one environment to perform simulation at the system level. There exist several modeling formalisms, each of which is adapted to certain kinds of problems. In the modeling formalism considered in this thesis, a model is represented by a set of Ordinary Differential Equations (ODEs) that describe the dynamics of the modeled system. The evolution of the simulation consists in numerically integrating the ODEs while minimizing the error.

The simulation of dynamical systems such as CPS can be accomplished in different ways according to the desired goal. In co-simulation, the different subsystems are described by models of equations and connected together to simulate the whole system on a computer. In this case, synchronized communications have to be ensured between the different models where each model must be able to detect and respond to events raised by the other models. Integrating heterogeneous models usually results in a complex and computationally expensive co-simulation which increases the demand of processing power. Consequently, a principal challenge of co-simulation is the question of how to reduce the execution time. As is well-known, increasing CPU frequency by means of silicon integration has reached its possible limits and semiconductor manufacturers switched in last years to building multi-core processors, i.e. integrating multiple processors into one chip allowing parallel processing on a single computer. Multi-core processors allow reducing the execution time of a program by partitioning it into a set of computational tasks and assigning a subset of tasks to each core to be processed in parallel.

In Hardware-in-the-Loop (HiL) simulation, physically available components, e.g. controller hardware, are connected to simulated models on a computer. The controller hardware runs the control algorithm (controller software) and is connected to the simulation computer via electronic interfaces. The goal here is to emulate the behavior of the real physical process so as to run the controller software under realistic conditions. The HiL approach is usually used to test the controller software on its final execution platform. However, the physically available component can instead be a part of the physical process. In HiL, two concepts of time have to be correctly meshed: the simulated time and the real-time. %The simulated time corresponds to the time needed to transfer data between models while the real-time corresponds to the time reference of the phsyciall available components. 
Achieving a correct meshing of the simulated time and the real-time defines a set of timing constraints applied on simulated models. These constraints have to be considered during the execution of the simulation. It is not always possible to satisfy these constraints, especially, on single-core processors. Performing HiL simulations on multi-core processors can enhance the opportunities of satisfying timing constraints which are infeasible on single-core processors.

\section{Objectives and Contributions}

There are two main research focuses in this thesis: acceleration of co-simulation and HiL co-simulation under real-time constraints on multi-core architectures. We are interested in co-simulations of CPS that are compliant with the FMI standard \cite{fmi:2014}. FMI facilitates the coupling of diverse models originating from different developers and tools. As already stated, a main problem of co-simulations is their expensive computational cost. Unfortunately, many simulation tools have single-core simulation kernels and do not take advantage of the computation power brought by multi-core architectures. Therefore, enabling parallel execution of computationally expensive co-simulations on multi-core processors is keenly sought by the developers and the users of simulation tools. In this context, we aim at developing appropriate algorithms to efficiently exploit the parallelism provided by multi-core processors in order to accelerate FMI co-simulations and possibly satisfy timing constraints of HiL co-simulations. Different approaches for parallelizing co-simulations are possible and have already been explored. In this thesis, we build on the existing solutions developed at IFP Energies nouvelles and seek to improve them. 

Developed at IFP Energies nouvelles, xMOD is a co-simulation and a virtual experimentation platform which allows mixing stand-alone and tool coupling co-simulations and optimizing complex models execution. The Refined CO-SIMulation (RCOSIM) approach \cite{benkhaled:2014} is the parallelization approach used in xMOD. It uses the information given by FMI about inputs and outputs relationships inside a model that is exported as a Functional Mock-up Unit (FMU). A model's FMU is a package that encapsulates an XML file containing, among other data, the definitions of the model's variables, and a library defining the equations of the model as C functions. Given these features, various execution possibilities can be realized, mainly by exploiting intra-model parallelism. The parallelization of co-simulation models on a multi-core processor can be seen as the following problem: find an allocation of the functions of the different models to the different cores and define an execution order, i.e. schedule the functions that are allocated to each core. When solving this problem, the utilization of the available cores has to be optimized in order to achieve the best acceleration. Using parallel computing terminology, the problem consists in finding a schedule for all the functions of the co-simulation on a multi-core processor. In this thesis, we continue the work on RCOSIM by addressing some of its limitations, presented below, in order to improve its performance and also to extend its use to different kinds of co-simulations.

In \cite{faure:2011} a set of rules is defined for propagating timing constraints from a physically available component to simulated models in a HiL co-simulation. It defines the constraints for each model of the co-simulation. In this thesis, we extend these rules to apply them on FMI compliant models. Furthermore, we propose multi-core scheduling algorithms to satisfy the defined timing constraints.

The contributions of this thesis are the following:

\begin{itemize}

\item \textbf{A dependence graph model for representing FMI co-simulations:} Our contributions consist in extending the dependence graph model of the RCOSIM approach to handle additional requirements and constraints as follows:

\begin{enumerate}

\item We extend the mono-rate dependence graph model to handle multi-rate FMI co-simulation. Such co-simulation involves FMUs that are assigned different communication step
sizes which define the data exchange rates of the FMUs. Such co-simulations cannot be handeled by RCOSIM because the dependence graph model fails to represent the different communication step sizes. We propose a transformation algorithm that transforms the dependence graph of a multi-rate FMI co-simulation. The result is a new graph that represents well the data exchange rates of the different FMUs.

\item We propose a method for handling mutual exclusion constraints between functions of a same FMU.  It is not possible to execute such functions in parallel because they share some resource, e.g. variables. The RCOSIM approach handles these constraints in a way that limits the attainable acceleration of the co-simulation. Our proposed solution results in a new graph that defines an order of execution for functions that are mutually exclusive. In order to obtain this new graph, we propose an acyclic orientation ILP formulation and heuristic. Our proposed solution outperforms RCOSIM.

\item We add to the dependence graph model real-time constraints to perform HiL FMI co-simulation when some models are replaced by there real counterparts that are physically available. Since real-time constraints are  imposed by the real parts on some inputs and outputs of the simulated models, we propose propagation algorithms that assign, according to the dependence, real-time constraints (release and deadline dates) to the nodes of the dependence graph. 

\newcounter{enumTemp}
\setcounter{enumTemp}{\theenumi}

\end{enumerate}

\item \textbf{Multi-core scheduling of FMI co-simulations:} We improve the scheduling heuristic used in the RCOSIM approach and propose other algorithms as follows:

\begin{enumerate}

\setcounter{enumi}{\theenumTemp}

\item We propose multi-core scheduling algorithms for the problem of co-simulation acceleration. We improve the multi-core scheduling heuristic used in RCOSIM by using profiled execution times and accounting for synchronization cost. Also, we propose an Integer Linear Programming (ILP) formulation. We compare the performances of the two algorithms for different sizes of dependence graphs and architectures.

\item We implemented a runtime scheduling solution using the Intel TBB library \cite{reinders:2007} for the acceleration of FMI co-simulation. We compared the performance of this solution with that of our proposed heuristic by applying both on an industrial use case.

\item We propose multi-core scheduling algorithms for the problem of HiL FMI co-simulations under real-time constraints. These algorithms consist in an ILP formulation and a heuristic. We compare the performances of the two algorithms for different sizes of dependence graphs and architectures.   

\end{enumerate}

\end{itemize}

\section{Thesis Outline}

The rest of this thesis is structured in six chapters. In Chapter \ref{ch:2-bkgnd} we give basic concepts and preliminaries related to the different domains that are involved in the thesis. In the first section, we present basic concepts of modeling and simulation. First we specify the type of systems that we are interested in. Then, we briefly present several modeling formalisms with an emphasis on differential equations and hybrid modeling. Next, we present the principle of numerical simulation before defining the concept of co-simulation. Co-simulation under real-time constraints is defined afterward. Finally, we review some of the most known tools for modeling and simulation, and real-time simulation.

In Chapter \ref{ch:3-state}, we give a detailed description of the research problem of this thesis. We start, in the first section, by giving an overview of the problem. In the second section, we present the RCOSIM approach that we aim at improving and extending in this thesis. Next, we list the limitations of RCOSIM. Finally, we present the open research issues and the objectives of the thesis in detail.

Chapter \ref{ch:4-accel} focuses on the first part of our contributions, i.e. a dependence graph model for representing FMI co-simulations. First, we present in detail the method of construction of the dependence graph of an FMI co-simulation, used in the RCOSIM approach. The next section is dedicated to multi-rate FMI co-simulation where we propose a transformation algorithm for multi-rate dependence graphs and give a small illustrative example. Then, we deal with the problem of handling mutual exclusion constraints. We give a detailed description of the problem and formulate it as an acyclic orientation problem. We propose a heuristic and and ILP formulation that perform the acyclic orientation of the dependence graph. In the last section, we describe the problem of propagating real-time constraints before detailing the proposed propagation algorithms. We give small examples to illustrate the algorithms.

In Chapter \ref{ch:5-sched}, we present our proposed scheduling heursitics and ILP formulations for both the acceleration of the co-simulation and the execution of HiL co-simulation under real-time constraints.

We evaluate our proposed solutions in Chapter \ref{ch:6-eval}. First, we propose a random generator of synthetic FMI co-simulations. Then we compare the performances of the heuristics and the ILP formulations for the acyclic orientation and the scheduling respectively. Finally, we evaluate our approach by applying it on an industrial use case. We compare its performance with RCOSIM and a runtime scheduling solution based on the Intel TBB library.

Chapter \ref{ch:7-concl} concludes this thesis. First, we give a summary of the objectives and the contributions of the thesis. Then, we present some perspectives for future work. 
