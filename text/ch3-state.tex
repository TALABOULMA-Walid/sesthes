\chapter{\label{ch:3-state}Problem Statement} 

\minitoc

In the foregoing part of the thesis, we presented the background of the involved research work. We presented preliminary concepts and state of the art review regarding the different topics that are involved in this thesis. In this chapter, we detail the research problem of the thesis. We give an overview of the adopted methods and explain the research problems that this thesis attempts to solve.

\section{Overview}

This thesis addresses the problem of parallel execution of co-simulation. We are interested in co-simulations that are compliant with the FMI standard. Both FMI for Model Exchange and FMI for Co-Simulation are of interest in this thesis. In particular, we focus on closed-source FMI co-simulations, i.e. co-simulations for which functions are provided as executable binaries and the source code is not accessible.

From a hardware standpoint, we target shared-memory architectures, more specifically multi-core architectures such as the ones found in desktop and laptop computers.

The research carried out in this thesis can be divided into two parts. The first part deals with accelerated co-simulation, i.e. the goal of the paralleization of the co-simulation is to accelerate its execution on multi-core architectures. In the second part, we focus on co-simulation under real-time constraints. In particular, we are interested in HiL (Hardware-in-the-Loop) co-simulation where a part of the co-simulation is replaced by its real counterpart that is physically available. 
The goal of this part is to satisfy the constraints for a real-time execution of the simulated part through parallelization on multi-core architectures.

\section{The RCOSIM Approach}

The contributions of this thesis constitute improvements to the RCOSIM approach briefly presented in section \ref{subsec:parsimaprr}. The RCOSIM approach is based on offline multi-core scheduling. First, it transforms the co-simulation FMU graph into a dependence graph with finer granularity. This process is detailed in Section \ref{sec:4-depgrph}. A multi-core list scheduling heuristic is then used to compute a schedule for this dependence graph, i.e. an allocation of its vertices, which represent functions, to the cores and an order of execution for the vertices that are allocated to each core. Each run of this schedule corresponds to the execution of one simulation step, i.e. update of the inputs, the outputs, and the state. Therefore, the execution of the co-simulation consists in running this schedule repeatedly until the number of desired simulation steps is reached.

In this thesis, we chose to build on the RCOSIM approach in order to achieve parallelization of FMI co-simulations for both accelerated and real-time execution. We did not use known parallel programming libraries for the following specific reasons. It is clear that MPI is not suitable for our goal since we target shared memory architectures whereas MPI is used to program distributed memory architectures. The other option is to use OpenMP or similar libraries which are adapted to shared-memory architectures. However, OpenMP is efficient especially in the case of data parallelism (e.g. loop parallelism) which is not apparent in the co-simulations that we target. In fact, since we do not have access to the source code of the functions, we can not perform parallelization of the functions code, e.g. solver function, by using OpenMP pragmas. We only have information about the co-simulation at the function level, i.e. the functions can only be called but their code cannot be accessed. It should be noted that libraries such as OpenMP and Intel TBB offer task programming features which can be used to execute multiple functions in parallel. Note that the code of each function is not parallelized, but two or more functions can be executed in parallel using this solution. However, they rely on online scheduling which may introduce high overhead and thus decreases the performance. In addition, given that information about dependence between functions is available and the execution times can be measured, we assume that offline scheduling is more efficient to achieve our goal.   

\section{RCOSIM Limitations}

Although RCOSIM resulted in interesting co-simulation speed-ups, it has some limitations that have to be considered in the parallelization problem in order to obtain better performances. We identify the following limitations of the RCOSIM approach:

\begin{enumerate}

\item So far, the multi-core scheduling heuristic uses empiric execution times of the different functions. By using realistic execution times, the multi-core execution of the co-simulation should be improved.

\item Only mono-rate co-simulations, i.e. where all the FMUs have the same communication step size, can be handled by RCOSIM. The equations of each FMU are integrated using a specific integration step size whereas its inputs and outputs are updated according to a specific communication step size which is a multiple of the integration step size. This is due to the lack of information about the communication step sizes in the dependence graph constructed by RCOSIM.

\item The FMI standard does not presently require that functions of the same FMU have to be thread-safe, i.e. they cannot be executed simultaneously as they may share a resource (e.g. variable) that might be corrupted if two operations try to use it at the same time. In other words, such functions have to be executed in strictly disjoint time intervals. This is tackled in RCOSIM by modifying the multi-core scheduling heuristic to always allocate the functions of a same FMU to the same core. Consequently, the search space of the scheduling heuristic is reduced, i.e. for a given function, if there is another function of the same FMU that has already been allocated to a specific core, it is allocated to this same core without the need to test it on the other cores. This restriction on the allocation possibilities limits the exploitation of the potential parallelism.

\item The multi-core scheduling heuristic used in RCOSIM is not real-time. In fact, RCOSIM targets only accelerated co-simulation and cannot be used for real-time co-simulation.

\item The dependence graph constructed by RCOSIM does not contain information about real-time constraints. Such information is necessary in order to be able to apply a real-time scheduling algorithm. For instance, it is needed to know if the execution of a given function has to be finished before a specific deadline.

\end{enumerate} 

\section{Open Research Issues and Thesis Objectives}

We aim in this thesis at proposing solutions to overcome the limitations of the RCOSIM approach presented in the previous section. Below, we detail the research questions that correspond to these limitations.

\subsubsection{Multi-rate FMU Co-simulation}

Most industrial applications are multi-rate, i.e. the different FMUs of the co-simulation are executed according to different communication step sizes. In order to parallelize a multi-rate co-simulation, it is needed to incorporate this information in the dependence graph model. Depending on how this information is added, the multi-core scheduling heuristic used in RCOSIM may need to be modified. Therefore, the research question related to this point is how should the dependence graph model be extended in order to allow RCOSIM to handle multi-rate co-simulation and what is the impact of such extension on the multi-core scheduling problem?

\subsubsection{Mutual Exclusion Constraints in FMU Co-simulation}

The non thread safe implementation of FMUs implies that at any instant during the execution of the co-simulation, one and only one function of the same FMU can be executed. Consequently, if the scheduling heuristic allocates two or more functions belonging to the same FMU to different cores, a mechanism that ensures these operations are executed in strictly different time intervals must be set up. We address the following questions: How can such mutual exclusion constraints be satisfied while optimizing the exploitation of the potential parallelism? Should the dependence graph model be extended or should these constraints be handled by the multi-core scheduling heuristic?

\subsubsection{Specification of Real-time Constraints for FMU Co-simulation}

In a co-simulation under real-time constraints, and more specifically HiL co-simulation, the physically available part periodically exchanges data with the simulated part, according to specific periods. The simulated part has to deliver data to the physically available components following these periods. Therefore, their execution has to be done within specific deadlines. Otherwise, the co-simulation fails. Given the periods of data exchange, we seek to define what are the constraints that have to be respected by the different FMUs in order to satisfy the real-time constraints. In other words, how should the executions of the different simulated FMUs be bounded so as to be able to deliver data in time to the real components at every period?

\subsubsection{Multi-core real-time scheduling for FMU Co-simulation}

Based on the defined real-time constraints, a multi-core scheduling algorithm is needed in order to allocate the functions of the simulated FMUs on the multi-core architecture in such a way to respect the real-time constraints. We are interested also in defining how the schedulability of a given co-simulation under real-time constraints on a given multi-core architecture can be tested?
%We are interested also in elaborating a schedulability test that determines if a given co-simulation can be executed on a specific multi-core architecture while respecting all the real-time constraints or not.  