\chapter{\label{ch:7-concl}Conclusion} 

\minitoc

This chapter concludes the thesis. First we give a summary of the work presented in the previous chapters. Second, we give some perspectives for future work.

\section{Summary}

The complexity of CPS is steadily increasing due to several factors. A lot of efforts is being made in industry as well as in academia in order to implement technologies and methods that respond to the requirements and challenges in the design of complex CPS. Co-simulation is increasingly being adopted as a system-level simulation in the context of CPS design thanks to its advantages over monolithic simulation. Strengths of co-simulation include easy upgrade, reuse, and exchange of models, improved computational performance compared to monolithic simulation, and allowing better intervention of experts at the subsystem level in multi-domain design projects. This being said, co-simulation faces a number of challenges that have to be addressed. This thesis constitutes a contribution towards solving some of these challenges.   

In this thesis, we are interested in the rising requirements on the computational performance of FMI co-simulation. We build on the work that was previously developed at IFP Energies nouvelles and aim at improving the existing methods. The focus of the thesis is on multi-core execution of co-simulation. In particular, there are two main goals for the research in this thesis. First, we aim at overcoming the limitations of the RCOSIM approach in order to allow the acceleration of different kinds of co-simulation. Second, we aim at extending the use of RCOSIM to co-simulation under real-timje constraints in the context of HiL. Below we summarize the contributions of this thesis.

In chapter \ref{ch:4-accel} we propose extensions to the operation graph model used in RCOSIM to represent the co-simulation. The first extension targets multi-rate co-simulation. We propose some rules for transforming a multi-rate operation graph into a mono-rate one in order to prepare its multi-core scheduling. Based on these rules, we propose an algorithm that performs this transformation.

The second extension consists in transforming the operation graph in order to handle mutual exclusion constraints between operations. First, the operation graph is transformed into a mixed graph by adding (non oriented) edges between mutually exclusive operations. Then, an acyclic orientation is computed for the mixed graph by assigning a direction to each edge. We propose two algorithm to perform the acyclic orientation: an ILP-based exact algorithm and a heuristic.

The last extension aims at completing the operation graph of a co-simulation under real-time constraints to enbale the application of a real-time multi-core scheduling algorithm. We focus on HiL co-simulation composed of a real and a simulated components. We propose methods for propagating real-time constraints imposed by inputs and outputs of the real component on gate operations of the simulated component. Such constraints are propagated to all the operation of the graph assigning a release and a deadline date to each operation. We propose two algorithms to perform the propagation of release and deadline constraints respectively.     

\section{Perspectives}