\chapter{\label{ch:7-concl}Conclusion} 

\minitoc

This chapter concludes the thesis. First we give a summary of the work presented in the previous chapters. Second, we give some perspectives for future work.

\section{Summary}

The complexity of CPS is steadily increasing due to several factors. A lot of efforts is being made in industry as well as in academia in order to implement technologies and methods that respond to the requirements and challenges in the design of complex CPS. Co-simulation is increasingly being adopted as a system-level simulation approach in the context of CPS design thanks to its advantages over monolithic simulation. Strengths of co-simulation include easy upgrade, reuse, and exchange of models, improved computational performance compared to monolithic simulation, and allowing better intervention of experts at the subsystem level in multi-domain design projects. This being said, co-simulation faces a number of challenges that have to be addressed. This thesis constitutes a contribution towards solving some of these challenges.   

In this thesis, we are interested in the rising requirements on the computational performance of FMI co-simulation. We build on the work that was previously developed at IFP Energies nouvelles and aim at improving the existing methods. The focus of the thesis is on multi-core execution of co-simulation. In particular, there are two main goals for the research in this thesis. First, we aim at overcoming the limitations of the RCOSIM approach in order to allow the acceleration of different kinds of co-simulation. Second, we aim at extending the use of RCOSIM to co-simulation under real-timje constraints in the context of HiL. Below we summarize the contributions of this thesis.

In Chapter \ref{ch:4-accel} we propose extensions to the operation graph model used in RCOSIM to represent the co-simulation. The first extension targets multi-rate co-simulation. We propose some rules for transforming a multi-rate operation graph into a mono-rate one in order to prepare its multi-core scheduling. Based on these rules, we propose an algorithm that performs this transformation.

The second extension consists in transforming the operation graph in order to handle mutual exclusion constraints between operations. First, the operation graph is transformed into a mixed graph by adding (non oriented) edges between mutually exclusive operations. Then, an acyclic orientation is computed for the mixed graph by assigning a direction to each edge. We propose two algorithm to perform the acyclic orientation: an ILP-based exact algorithm and a heuristic.

The last extension aims at completing the operation graph of a co-simulation under real-time constraints to enbale the application of a real-time multi-core scheduling algorithm. We focus on HiL co-simulation composed of a real and a simulated components. We propose methods for propagating real-time constraints imposed by inputs and outputs of the real component on gate operations of the simulated component. Such constraints are propagated to all the operation of the graph assigning a release and a deadline date to each operation. We propose two algorithms to perform the propagation of release and deadline constraints respectively.

In Chapter \ref{ch:5-sched} we propose multi-core scheduling algorithms. First, we focus on co-simulation acceleration. For this we propose two multi-core scheduling algorithms. The first algorithm is an ILP-based exact algorithm and the second one is a list scheduling heuristic. The schedule is computed using either of these algorithms over the hyperstep. During execution, this schedule is executed repeatedly.

Second, we propose algorithms for multi-core scheduling of co-simulation under real-time constraints. Similarly, we propose an ILP-based exact algorithm and a list scheduling heuristic. Also, we propose a simple technique to study the schedulability of the operation graph and determining a periodic pattern of the schedule.

Finally, in Chapter \ref{ch:6-eval}, we evaluate our proposed approach. First, we propose a random generator of operation graphs. We use this graph to generate a large number of synthetic operation graphs of different sizes and structures and with different attributes. For co-simulation under real-time constraints, the generator generates also a random number of inputs and outputs of the real component and connect it to the operation graph. 

We evaluate the performances of our proposed ILP-based exact algorithms and heuristics for the acyclic orientation, scheduling for co-simulation acceleration, and real-time scheduling respectively. The obtained results show the efficiency of our heuristics. While the proposed ILP algorithms give optimal results for small operation graphs they suffer from intractable execution times. Our proposed heuristics, on the other hand, give acceptable results within acceptable execution times.

Last, we validate our approach for co-simulation acceleration against an industrial use case. The obtained results show the improvements made thanks to using multi-rate co-simulation and also using the acyclic orientation to handle mutual exclusion constraints. In addition, we compare our approach with a runtime (online) scheduling approach. Our approach outperforms it which consolidates our choice of adopting an offline approach.               

\section{Perspectives}

We present below some possible research directions for future work.

\subsubsection{Grain Size Determination}

Our proposed approach relies on partitioning the co-simulation into operations that are scheduled in parallel. The size of the operations, referred to as \textit{grain size} in parallel computing terminology, may have an important impact on the achievable performance. In fact, there is a tradeoff between the grain size and the overhead of scheduling.

Addressing the problem of grain size determination in operation graphs should strengthen our proposed approach. There exist some approaches in the literature such as \cite{kruatrachue:1988} that can be tested on operation graphs at first. Then, the use of graph clustering algorithms can be explored in order to address this problem. 

\subsubsection{Real-time Schedulability Analysis}

In our proposed approach for scheduling co-simulation under real-time constraints, we used a schedulability analysis based on simulation. In addition, we do not determine the schedulability interval before computing the schedule. Moreover, the existing results on the schedulability interval are very pessimistic. Therefore, it is very important to derive an analytic schedulability condition. Alternatively, finding a more optimistic schedulability interval than the existing results would represent a major enhancement.  